\documentclass[12pt]{article}

\usepackage{float}
\usepackage{morefloats}

\usepackage{graphicx}
\graphicspath{ {./img/} }

\usepackage[utf8]{inputenc}
\usepackage[italian]{babel}
\usepackage[T1]{fontenc}

\usepackage{pgf,pgffor}

\usepackage{xstring}

\usepackage{hyperref}

\title{covid19-report}
\author{Enrico Cambiaso}
\date{\today}

\newcommand{\figureswidth}{8cm}

\newcommand{\fieldsitalia}{ricoverati_con_sintomi,terapia_intensiva,totale_ospedalizzati,totale_positivi,nuovi_positivi,nuovi_tamponi,deceduti,nuovipositivi_su_nuovitamponi}
\newcommand{\fieldsitaliatext}{Ricoverati con sintomi,Terapia intensiva,Totale ospedalizzati,Totale positivi,Nuovi positivi,Nuovi tamponi,Deceduti,Rapporto nuovi positivi su nuovi tamponi}

\newcommand{\fieldsregioni}{ricoverati_con_sintomi,terapia_intensiva,totale_ospedalizzati,totale_positivi,nuovi_positivi,nuovi_tamponi,deceduti,nuovipositivi_su_nuovitamponi}
\newcommand{\fieldsregionitext}{Ricoverati con sintomi,Terapia intensiva,Totale ospedalizzati,Totale positivi,Nuovi positivi,Tamponi,Deceduti,Rapporto nuovi positivi su nuovi tamponi}

\newcommand{\fieldsprovince}{totale_casi}
\newcommand{\fieldsprovincetext}{Totale casi}

\newcommand{\regions}{abruzzo,basilicata,calabria,campania,emiliaromagna,friuliveneziagiulia,lazio,liguria,lombardia,marche,molise,pabolzano,patrento,piemonte,puglia,sardegna,sicilia,toscana,umbria,valledaosta,veneto}

\newcommand{\ondate}{1,2,3}

%\newcommand{\province}{}

\usepackage{xparse}
\ExplSyntaxOn
\NewExpandableDocumentCommand{\GetListMember}{mm}
 {
  \bool_lazy_and:nnTF { \tl_if_single_p:n { #1 } } { \token_if_cs_p:N #1 }
   {
    \clist_item:Nn #1 { #2 }
   }
   {
    \clist_item:nn { #1 } { #2 }
   }
 }
\ExplSyntaxOff

\newcommand{\regionstext}{Abruzzo,Basilicata,Calabria,Campania,Emilia Romagna,Friuli Venezia Giulia,Lazio,Liguria,Lombardia,Marche,Molise,P.A. Bolzano,P.A. Trento,Piemonte,Puglia,Sardegna,Sicilia,Toscana,Umbria,Valle D'Aosta,Veneto}

\setcounter{tocdepth}{2}

\begin{document}

\begin{titlepage}
\maketitle
\thispagestyle{empty}
\vfill
{\centering \small \href{http://github.com/auino/covid19-report}{github.com/auino/covid19-report} \par}
\end{titlepage}

\tableofcontents

\newpage

\begin{center}
\emph{This project is supposed to be used for Italian users. Therefore, the whole content of the generated documents is in Italian language.}
\end{center}

\section{Introduzione}

Questo progetto ha lo scopo di raggruppare i dati disponibili pubblicamente da diverse fonti e fornire un report unificato che analizzi e processi i dati per fornire una visione complessiva relativamente all'andamento del Covid-19 in Italia.

Maggiori informazioni sul software alla base di questo report sono disponibili sulla relativa pagina \href{https://github.com}{GitHub}: \href{https://github.com/auino/covid19-report}{github.com/auino/covid19-report}.

\subsection{Fonti utilizzate}

A seguire la lista di fonti utilizzate:

\begin{itemize}
	\item \href{https://github.com/pcm-dpc/COVID-19}{pcm-dpc/COVID-19} per il recupero di dati a livello nazionale, regionale e provinciale
	\item \href{https://github.com/italia/covid19-opendata-vaccini}{italia/covid19-opendata-vaccini} per il recupero dei dati relativi alla somministrazione dei vaccini
	\item \href{https://github.com/CloudItaly/Indice-RT}{CloudItaly/Indice-RT} per il recupero dei dati regionali relativi all'indice $R_t$ misurato
	\item \href{http://news.google.com}{Google News} per ricerca di notizie in tema Covid-19
\end{itemize}

\subsection{Estensioni}

Per eventuali estensioni, si rimanda alla pagina \href{https://github.com}{GitHub} del progetto:\\
\href{https://github.com/auino/covid19-report}{github.com/auino/covid19-report}.

\newpage

\section{Andamento nazionale}

\subsection{Dati relativi alla giornata odierna}

\input{summary_italia}

Vengono a seguire riportate le ultime informazioni a livello nazionale:
\begin{itemize}
	\item Ricoverati con sintomi: $\italiaricoveraticonsintomi$
	\item Soggetti in terapia intensiva: $\italiaterapiaintensiva$
	\item Totale ospedalizzati: $\italiatotaleospedalizzati$
	\item Soggetti in isolamento domiciliare: $\italiaisolamentodomiciliare$
	\item Totale positivi: $\italiatotalepositivi$
	\item Variazione sul numero totale dei positivi: $\italiavariazionetotalepositivi$
	\item Nuovi positivi: $\italianuovipositivi$
	\item Soggetti dimessi guariti: $\italiadimessiguariti$
	\item Soggetti deceduti: $\italiadeceduti$
	\item Casi da sospetto diagnostico: $\italiacasidasospettodiagnostico$
	\item Casi da screening: $\italiacasidascreening$
	\item Totale casi: $\italiatotalecasi$
	\item Tamponi: $\italiatamponi$
	\item Rapporto nuovi positivi su nuovi tamponi: $\italianuovipositivisunuovitamponi$
	\item Nuovi tamponi: $\italianuovitamponi$
	\item Casi testati: $\italiacasitestati$
	\item Ingressi in terapia intensiva: $\italiaingressiterapiaintensiva$
	\item Totale positivi al test molecolare: $\italiatotalepositivitestmolecolare$
	\item Totale positivi al test antigenico rapido: $\italiatotalepositivitestantigenicorapido$
	\item Tamponi del test molecolare: $\italiatamponitestmolecolare$
	\item Tamponi del test antigenico rapido: $\italiatamponitestantigenicorapido$
\end{itemize}

\subsection{Ultime notizie}

A seguire una lista di notizie odierne sull'argomento, recuperate da \href{http://news.google.it}{Google News}.
\begin{itemize}
	\input{news_italia}
\end{itemize}

\subsection{Ultimi dati su indice $R_t$}

\begin{center}
\begin{tabular}{ |c|c|c|} 
	\hline
	\bfseries{Regione} & \bfseries{Indice $R_t$} \\ \hline
	\input{italia-rt-table.tex}
	\hline
\end{tabular}
\end{center}

\input{italia-rt.tex}

Dati aggiornati al \italiaindicertdata.

\subsection{Ultimi dati su rapporto nuovi positivi su nuovi tamponi}

\begin{center}
\begin{tabular}{ |c|c|c|} 
	\hline
	\bfseries{Regione} & \bfseries{Rapporto misurato (\%)} \\ \hline
	\input{italia-nuovipositivisunuovitamponi-table.tex}
	\hline
\end{tabular}
\end{center}

\subsection{Grafici di monitoraggio}

\foreach \x [count=\xi] in \fieldsitalia 
{
	\begin{figure}[H]
	\centering
	\includegraphics[width=\figureswidth]{italia-\x.png}
	\label{italia-\x}
		\caption{\GetListMember{\fieldsitaliatext}{\xi} in Italia}
	\end{figure}

	\foreach \o [count=\oi] in \ondate 
	{
		\begin{figure}[H]
		\centering
		\includegraphics[width=\figureswidth]{italia-\x_\o.png}
		\label{italia-\x-\o}
			\caption{\GetListMember{\fieldsitaliatext}{\xi} in Italia (ondata \o)}
		\end{figure}
	}
}

\newpage

\section{Andamento per regione}

\foreach \r [count=\ri] in \regions 
{
	\subsection{\GetListMember{\regionstext}{\ri}}

	\subsubsection*{Dati relativi alla giornata odierna}

	\input{summary_regione_\r}

	Vengono a seguire riportate le ultime informazioni a livello regionale, per la regione \r:
	\begin{itemize}
			\item Ricoverati con sintomi: $\regionericoveraticonsintomi$
			\item Soggetti in terapia intensiva: $\regioneterapiaintensiva$
			\item Totale ospedalizzati: $\regionetotaleospedalizzati$
			\item Soggetti in isolamento domiciliare: $\regioneisolamentodomiciliare$
			\item Totale positivi: $\regionetotalepositivi$
			\item Variazione sul numero totale dei positivi: $\regionevariazionetotalepositivi$
			\item Nuovi positivi: $\regionenuovipositivi$
			\item Soggetti dimessi guariti: $\regionedimessiguariti$
			\item Soggetti deceduti: $\regionedeceduti$
			\item Totale casi: $\regionetotalecasi$
			\item Tamponi: $\regionetamponi$
			\item Rapporto nuovi positivi su nuovi tamponi: $\regionenuovipositivisunuovitamponi$
			\item Nuovi tamponi: $\regionenuovitamponi$
			\item Casi testati: $\regionecasitestati$
			\item Ingressi in terapia intensiva: $\regioneingressiterapiaintensiva$
			\item Totale positivi al test molecolare: $\regionetotalepositivitestmolecolare$
			\item Totale positivi al test antigenico rapido: $\regionetotalepositivitestantigenicorapido$
			\item Tamponi del test molecolare: $\regionetamponitestmolecolare$
			\item Tamponi del test antigenico rapido: $\regionetamponitestantigenicorapido$
			\item Dati di somministrazione di vaccino, aggiornati al \regioneultimoaggiornamento:
			\begin{itemize}
				\item Dosi somministrate: $\regionedosisomministrate$
				\item Percentuale di somministrazione: $\regionepercentualesomministrazione\%$
			\end{itemize}
	\end{itemize}

	\subsubsection*{Grafici di monitoraggio}

	\foreach \x [count=\xi] in \fieldsregioni 
	{
		\begin{figure}[H]
		\centering
		\includegraphics[width=\figureswidth]{\r-\x.png}
		\label{\r-\x}
		\caption{\GetListMember{\fieldsregionitext}{\xi} in \GetListMember{\regionstext}{\ri}}
		\end{figure}

		\foreach \o [count=\oi] in \ondate 
		{
			\begin{figure}[H]
			\centering
			\includegraphics[width=\figureswidth]{\r-\x_\o.png}
			\label{\r-\x-\o}
				\caption{\GetListMember{\fieldsregionitext}{\xi} in \GetListMember{\regionstext}{\ri} (ondata \o)}
			\end{figure}
		}
	}

	\input{province_\r}

	\foreach \p [count=\pi] in \regioniprovince 
	{
		\subsubsection{\GetListMember{\regioniprovincetext}{\pi}}
		
		\subsubsection*{Grafici di monitoraggio}

		\foreach \xx [count=\xxi] in \fieldsprovince 
		{
			\begin{figure}[H]
				\centering
				\includegraphics[width=\figureswidth]{\p-\xx.png}
				\label{\p-\xx}
				\caption{\GetListMember{\fieldsprovincetext}{\xxi} in \GetListMember{\regioniprovincetext}{\pi}}
			\end{figure}

			\foreach \o [count=\oi] in \ondate 
			{
				\begin{figure}[H]
				\centering
				\includegraphics[width=\figureswidth]{\p-\xx_\o.png}
				\label{\p-\xx-\o}
					\caption{\GetListMember{\fieldsprovincetext}{\xxi} in \GetListMember{\regioniprovincetext}{\pi} (ondata \o)}
				\end{figure}
			}
		}
	}

	\newpage
}

%\section{Conclusioni}

%TODO

\end{document}